\documentclass{beamer}

\title{AEP vs. Polar Code}

\begin{document}

\begin{frame}
\titlepage
\end{frame}

\begin{frame}
	\tableofcontents
\end{frame}


%%%%%%%%%%%%%%%%%%%%%%%%%%%%%%%%%%%%%%%%%%%%%%%%%%%%%%%%%%
%%%%%%%%%%%%%%%%%%%%%%%%%%%%%%%%%%%%%%%%%%%%%%%%%%%%%%%%%%
%%%%%%%%%%%%%%%%%%%%%%%%%%%%%%%%%%%%%%%%%%%%%%%%%%%%%%%%%%

%%%%%%%%%%%%%%%%%%%%%%%%%%%%%%%%%
\begin{frame}
\frametitle{Ideas}
\begin{itemize}
\item Use AEP/entropy rate to explain channel polarization

\end{itemize}
\end{frame}

%%%%%%%%%%%%%%%%%%%%%%%%%%%%%%%%%
\begin{frame}
\frametitle{Channel Polarization}
\begin{itemize}
\item Two inputs:
\begin{align*}
I(U_1,U_2;Y_1,Y_2)
  &= I(U_1;Y_1,Y_2) + I(U_2;Y_1,Y_2|U_1) \\
  &= I(U_1;Y_1,Y_2) + \left( I(U_2;Y_1,Y_2,U_1) - I(U_2;U_1) \right) \\
  &= \color{green}{I(U_1;Y_1,Y_2)} + \color{blue}{I(U_2;Y_1,Y_2,U_1)} \\
  &= \color{green}{I(W^-)} + \color{blue}{I(W^+)}
\end{align*}

\item Using the chain rule:
\begin{align*}
I(U_1^n;Y_1^n)
  &= \sum_{i=1}^n I(U_i;Y_1^n|U_1^{i-1}) \\
  &= \sum_{i=1}^n \left[ I(U_i;Y_1^n,U_1^{i-1}) - I(U_i;U_1^{i-1}) \right] \\
  &= \sum_{i=1}^n I(U_i;Y_1^n,U_1^{i-1})
\end{align*}

\end{itemize}
\end{frame}

%%%%%%%%%%%%%%%%%%%%%%%%%%%%%%%%%%%%%%%%%%%%%%%%%%%%%%%%%%
%%%%%%%%%%%%%%%%%%%%%%%%%%%%%%%%%%%%%%%%%%%%%%%%%%%%%%%%%%
%%%%%%%%%%%%%%%%%%%%%%%%%%%%%%%%%%%%%%%%%%%%%%%%%%%%%%%%%%


\end{document}